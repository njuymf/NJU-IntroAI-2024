\documentclass[a4paper, 12pt]{article}  
\usepackage{amsmath}  
\usepackage{amsfonts}  
\usepackage{graphicx}  
\usepackage[UTF8]{ctex} % 支持中文  
\usepackage{hyperref}  
\usepackage{geometry}  
\usepackage{tikz}  
\usepackage{ulem}  
\usepackage{listings}  
\usepackage{float}  
\setlength{\textfloatsep}{5pt} % 减少图与文本之间的距离  
\usetikzlibrary{automata, positioning}  
\geometry{margin=1in}  


\title{作业4实验报告}  
\author{231240002余孟凡 231240002@smail.nju.edu.cn\\
	南京大学 计算机科学与技术系, 南京 210093}  
\date{}  

\begin{document}  
	
	\maketitle  
	
	\begin{abstract}  

	\end{abstract}  
	
	\textbf{关键词:}   Alpha GO
	
	\section{阅读论文}
	
	公式其实不是太懂,但大概流程是:
	
	\begin{enumerate}
		\item 首先,AlphaGo训练了一个监督学习(SL)策略网络,目标是预测人类专家在特定棋盘状态下的最佳移动。这个网络由多个卷积层组成,输入为棋盘状态的图像表示,输出为每个合法移动的概率分布。
		\item 在监督学习之后,AlphaGo通过自我对弈来进一步优化策略网络。使用强化学习(RL)策略网络,AlphaGo与之前版本的策略网络进行对弈,优化目标是赢得比赛而非仅仅提高预测准确性。
		\item AlphaGo还训练了一个价值网络,用于评估棋盘状态的胜率。这个网络的结构与策略网络相似,但输出的是一个标量值,表示当前状态的预期结果。
		\item AlphaGo结合了策略网络和价值网络与MCTS算法。MCTS通过模拟多次游戏来评估每个状态的价值,并在搜索树中选择最优动作。
	\end{enumerate}
	
	\section{阅读代码}
	
	从最外层代码开始看起:
	
	\textbf{rl\_loop.py }:创建两个随机策略的代理 agents,使用 RandomAgent 类。然后进行对弈,在每局对弈中,重置环境 env.reset(),获取初始状态 time\_step。在每一步中,调用对应代理的 step 方法选择动作 action\_list。环境执行该动作 env.step(action\_list),返回新的状态 time\_step。
	
	从这个函数的打印我们可以很直观的看到棋盘情况,并修改flatten\_board\_state观察一维和二维棋盘。
	
	\textbf{dqn\_vs\_random\_demo.py}

	
		
	
	\bibliographystyle{plain}  
	\bibliography{references}  
	
	(Mastering the game of Go with deep neural networks and tree search)
	
	(Technion – Israel Institute of Technology 	Project: Mini Alpha Go )
		
	
\end{document}  